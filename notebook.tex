
% Default to the notebook output style

    


% Inherit from the specified cell style.




    
\documentclass[11pt]{article}

    
    
    \usepackage[T1]{fontenc}
    % Nicer default font (+ math font) than Computer Modern for most use cases
    \usepackage{mathpazo}

    % Basic figure setup, for now with no caption control since it's done
    % automatically by Pandoc (which extracts ![](path) syntax from Markdown).
    \usepackage{graphicx}
    % We will generate all images so they have a width \maxwidth. This means
    % that they will get their normal width if they fit onto the page, but
    % are scaled down if they would overflow the margins.
    \makeatletter
    \def\maxwidth{\ifdim\Gin@nat@width>\linewidth\linewidth
    \else\Gin@nat@width\fi}
    \makeatother
    \let\Oldincludegraphics\includegraphics
    % Set max figure width to be 80% of text width, for now hardcoded.
    \renewcommand{\includegraphics}[1]{\Oldincludegraphics[width=.8\maxwidth]{#1}}
    % Ensure that by default, figures have no caption (until we provide a
    % proper Figure object with a Caption API and a way to capture that
    % in the conversion process - todo).
    \usepackage{caption}
    \DeclareCaptionLabelFormat{nolabel}{}
    \captionsetup{labelformat=nolabel}

    \usepackage{adjustbox} % Used to constrain images to a maximum size 
    \usepackage{xcolor} % Allow colors to be defined
    \usepackage{enumerate} % Needed for markdown enumerations to work
    \usepackage{geometry} % Used to adjust the document margins
    \usepackage{amsmath} % Equations
    \usepackage{amssymb} % Equations
    \usepackage{textcomp} % defines textquotesingle
    % Hack from http://tex.stackexchange.com/a/47451/13684:
    \AtBeginDocument{%
        \def\PYZsq{\textquotesingle}% Upright quotes in Pygmentized code
    }
    \usepackage{upquote} % Upright quotes for verbatim code
    \usepackage{eurosym} % defines \euro
    \usepackage[mathletters]{ucs} % Extended unicode (utf-8) support
    \usepackage[utf8x]{inputenc} % Allow utf-8 characters in the tex document
    \usepackage{fancyvrb} % verbatim replacement that allows latex
    \usepackage{grffile} % extends the file name processing of package graphics 
                         % to support a larger range 
    % The hyperref package gives us a pdf with properly built
    % internal navigation ('pdf bookmarks' for the table of contents,
    % internal cross-reference links, web links for URLs, etc.)
    \usepackage{hyperref}
    \usepackage{longtable} % longtable support required by pandoc >1.10
    \usepackage{booktabs}  % table support for pandoc > 1.12.2
    \usepackage[inline]{enumitem} % IRkernel/repr support (it uses the enumerate* environment)
    \usepackage[normalem]{ulem} % ulem is needed to support strikethroughs (\sout)
                                % normalem makes italics be italics, not underlines
    

    
    
    % Colors for the hyperref package
    \definecolor{urlcolor}{rgb}{0,.145,.698}
    \definecolor{linkcolor}{rgb}{.71,0.21,0.01}
    \definecolor{citecolor}{rgb}{.12,.54,.11}

    % ANSI colors
    \definecolor{ansi-black}{HTML}{3E424D}
    \definecolor{ansi-black-intense}{HTML}{282C36}
    \definecolor{ansi-red}{HTML}{E75C58}
    \definecolor{ansi-red-intense}{HTML}{B22B31}
    \definecolor{ansi-green}{HTML}{00A250}
    \definecolor{ansi-green-intense}{HTML}{007427}
    \definecolor{ansi-yellow}{HTML}{DDB62B}
    \definecolor{ansi-yellow-intense}{HTML}{B27D12}
    \definecolor{ansi-blue}{HTML}{208FFB}
    \definecolor{ansi-blue-intense}{HTML}{0065CA}
    \definecolor{ansi-magenta}{HTML}{D160C4}
    \definecolor{ansi-magenta-intense}{HTML}{A03196}
    \definecolor{ansi-cyan}{HTML}{60C6C8}
    \definecolor{ansi-cyan-intense}{HTML}{258F8F}
    \definecolor{ansi-white}{HTML}{C5C1B4}
    \definecolor{ansi-white-intense}{HTML}{A1A6B2}

    % commands and environments needed by pandoc snippets
    % extracted from the output of `pandoc -s`
    \providecommand{\tightlist}{%
      \setlength{\itemsep}{0pt}\setlength{\parskip}{0pt}}
    \DefineVerbatimEnvironment{Highlighting}{Verbatim}{commandchars=\\\{\}}
    % Add ',fontsize=\small' for more characters per line
    \newenvironment{Shaded}{}{}
    \newcommand{\KeywordTok}[1]{\textcolor[rgb]{0.00,0.44,0.13}{\textbf{{#1}}}}
    \newcommand{\DataTypeTok}[1]{\textcolor[rgb]{0.56,0.13,0.00}{{#1}}}
    \newcommand{\DecValTok}[1]{\textcolor[rgb]{0.25,0.63,0.44}{{#1}}}
    \newcommand{\BaseNTok}[1]{\textcolor[rgb]{0.25,0.63,0.44}{{#1}}}
    \newcommand{\FloatTok}[1]{\textcolor[rgb]{0.25,0.63,0.44}{{#1}}}
    \newcommand{\CharTok}[1]{\textcolor[rgb]{0.25,0.44,0.63}{{#1}}}
    \newcommand{\StringTok}[1]{\textcolor[rgb]{0.25,0.44,0.63}{{#1}}}
    \newcommand{\CommentTok}[1]{\textcolor[rgb]{0.38,0.63,0.69}{\textit{{#1}}}}
    \newcommand{\OtherTok}[1]{\textcolor[rgb]{0.00,0.44,0.13}{{#1}}}
    \newcommand{\AlertTok}[1]{\textcolor[rgb]{1.00,0.00,0.00}{\textbf{{#1}}}}
    \newcommand{\FunctionTok}[1]{\textcolor[rgb]{0.02,0.16,0.49}{{#1}}}
    \newcommand{\RegionMarkerTok}[1]{{#1}}
    \newcommand{\ErrorTok}[1]{\textcolor[rgb]{1.00,0.00,0.00}{\textbf{{#1}}}}
    \newcommand{\NormalTok}[1]{{#1}}
    
    % Additional commands for more recent versions of Pandoc
    \newcommand{\ConstantTok}[1]{\textcolor[rgb]{0.53,0.00,0.00}{{#1}}}
    \newcommand{\SpecialCharTok}[1]{\textcolor[rgb]{0.25,0.44,0.63}{{#1}}}
    \newcommand{\VerbatimStringTok}[1]{\textcolor[rgb]{0.25,0.44,0.63}{{#1}}}
    \newcommand{\SpecialStringTok}[1]{\textcolor[rgb]{0.73,0.40,0.53}{{#1}}}
    \newcommand{\ImportTok}[1]{{#1}}
    \newcommand{\DocumentationTok}[1]{\textcolor[rgb]{0.73,0.13,0.13}{\textit{{#1}}}}
    \newcommand{\AnnotationTok}[1]{\textcolor[rgb]{0.38,0.63,0.69}{\textbf{\textit{{#1}}}}}
    \newcommand{\CommentVarTok}[1]{\textcolor[rgb]{0.38,0.63,0.69}{\textbf{\textit{{#1}}}}}
    \newcommand{\VariableTok}[1]{\textcolor[rgb]{0.10,0.09,0.49}{{#1}}}
    \newcommand{\ControlFlowTok}[1]{\textcolor[rgb]{0.00,0.44,0.13}{\textbf{{#1}}}}
    \newcommand{\OperatorTok}[1]{\textcolor[rgb]{0.40,0.40,0.40}{{#1}}}
    \newcommand{\BuiltInTok}[1]{{#1}}
    \newcommand{\ExtensionTok}[1]{{#1}}
    \newcommand{\PreprocessorTok}[1]{\textcolor[rgb]{0.74,0.48,0.00}{{#1}}}
    \newcommand{\AttributeTok}[1]{\textcolor[rgb]{0.49,0.56,0.16}{{#1}}}
    \newcommand{\InformationTok}[1]{\textcolor[rgb]{0.38,0.63,0.69}{\textbf{\textit{{#1}}}}}
    \newcommand{\WarningTok}[1]{\textcolor[rgb]{0.38,0.63,0.69}{\textbf{\textit{{#1}}}}}
    
    
    % Define a nice break command that doesn't care if a line doesn't already
    % exist.
    \def\br{\hspace*{\fill} \\* }
    % Math Jax compatability definitions
    \def\gt{>}
    \def\lt{<}
    % Document parameters
    \title{titanic\_survival\_exploration}
    
    
    

    % Pygments definitions
    
\makeatletter
\def\PY@reset{\let\PY@it=\relax \let\PY@bf=\relax%
    \let\PY@ul=\relax \let\PY@tc=\relax%
    \let\PY@bc=\relax \let\PY@ff=\relax}
\def\PY@tok#1{\csname PY@tok@#1\endcsname}
\def\PY@toks#1+{\ifx\relax#1\empty\else%
    \PY@tok{#1}\expandafter\PY@toks\fi}
\def\PY@do#1{\PY@bc{\PY@tc{\PY@ul{%
    \PY@it{\PY@bf{\PY@ff{#1}}}}}}}
\def\PY#1#2{\PY@reset\PY@toks#1+\relax+\PY@do{#2}}

\expandafter\def\csname PY@tok@w\endcsname{\def\PY@tc##1{\textcolor[rgb]{0.73,0.73,0.73}{##1}}}
\expandafter\def\csname PY@tok@c\endcsname{\let\PY@it=\textit\def\PY@tc##1{\textcolor[rgb]{0.25,0.50,0.50}{##1}}}
\expandafter\def\csname PY@tok@cp\endcsname{\def\PY@tc##1{\textcolor[rgb]{0.74,0.48,0.00}{##1}}}
\expandafter\def\csname PY@tok@k\endcsname{\let\PY@bf=\textbf\def\PY@tc##1{\textcolor[rgb]{0.00,0.50,0.00}{##1}}}
\expandafter\def\csname PY@tok@kp\endcsname{\def\PY@tc##1{\textcolor[rgb]{0.00,0.50,0.00}{##1}}}
\expandafter\def\csname PY@tok@kt\endcsname{\def\PY@tc##1{\textcolor[rgb]{0.69,0.00,0.25}{##1}}}
\expandafter\def\csname PY@tok@o\endcsname{\def\PY@tc##1{\textcolor[rgb]{0.40,0.40,0.40}{##1}}}
\expandafter\def\csname PY@tok@ow\endcsname{\let\PY@bf=\textbf\def\PY@tc##1{\textcolor[rgb]{0.67,0.13,1.00}{##1}}}
\expandafter\def\csname PY@tok@nb\endcsname{\def\PY@tc##1{\textcolor[rgb]{0.00,0.50,0.00}{##1}}}
\expandafter\def\csname PY@tok@nf\endcsname{\def\PY@tc##1{\textcolor[rgb]{0.00,0.00,1.00}{##1}}}
\expandafter\def\csname PY@tok@nc\endcsname{\let\PY@bf=\textbf\def\PY@tc##1{\textcolor[rgb]{0.00,0.00,1.00}{##1}}}
\expandafter\def\csname PY@tok@nn\endcsname{\let\PY@bf=\textbf\def\PY@tc##1{\textcolor[rgb]{0.00,0.00,1.00}{##1}}}
\expandafter\def\csname PY@tok@ne\endcsname{\let\PY@bf=\textbf\def\PY@tc##1{\textcolor[rgb]{0.82,0.25,0.23}{##1}}}
\expandafter\def\csname PY@tok@nv\endcsname{\def\PY@tc##1{\textcolor[rgb]{0.10,0.09,0.49}{##1}}}
\expandafter\def\csname PY@tok@no\endcsname{\def\PY@tc##1{\textcolor[rgb]{0.53,0.00,0.00}{##1}}}
\expandafter\def\csname PY@tok@nl\endcsname{\def\PY@tc##1{\textcolor[rgb]{0.63,0.63,0.00}{##1}}}
\expandafter\def\csname PY@tok@ni\endcsname{\let\PY@bf=\textbf\def\PY@tc##1{\textcolor[rgb]{0.60,0.60,0.60}{##1}}}
\expandafter\def\csname PY@tok@na\endcsname{\def\PY@tc##1{\textcolor[rgb]{0.49,0.56,0.16}{##1}}}
\expandafter\def\csname PY@tok@nt\endcsname{\let\PY@bf=\textbf\def\PY@tc##1{\textcolor[rgb]{0.00,0.50,0.00}{##1}}}
\expandafter\def\csname PY@tok@nd\endcsname{\def\PY@tc##1{\textcolor[rgb]{0.67,0.13,1.00}{##1}}}
\expandafter\def\csname PY@tok@s\endcsname{\def\PY@tc##1{\textcolor[rgb]{0.73,0.13,0.13}{##1}}}
\expandafter\def\csname PY@tok@sd\endcsname{\let\PY@it=\textit\def\PY@tc##1{\textcolor[rgb]{0.73,0.13,0.13}{##1}}}
\expandafter\def\csname PY@tok@si\endcsname{\let\PY@bf=\textbf\def\PY@tc##1{\textcolor[rgb]{0.73,0.40,0.53}{##1}}}
\expandafter\def\csname PY@tok@se\endcsname{\let\PY@bf=\textbf\def\PY@tc##1{\textcolor[rgb]{0.73,0.40,0.13}{##1}}}
\expandafter\def\csname PY@tok@sr\endcsname{\def\PY@tc##1{\textcolor[rgb]{0.73,0.40,0.53}{##1}}}
\expandafter\def\csname PY@tok@ss\endcsname{\def\PY@tc##1{\textcolor[rgb]{0.10,0.09,0.49}{##1}}}
\expandafter\def\csname PY@tok@sx\endcsname{\def\PY@tc##1{\textcolor[rgb]{0.00,0.50,0.00}{##1}}}
\expandafter\def\csname PY@tok@m\endcsname{\def\PY@tc##1{\textcolor[rgb]{0.40,0.40,0.40}{##1}}}
\expandafter\def\csname PY@tok@gh\endcsname{\let\PY@bf=\textbf\def\PY@tc##1{\textcolor[rgb]{0.00,0.00,0.50}{##1}}}
\expandafter\def\csname PY@tok@gu\endcsname{\let\PY@bf=\textbf\def\PY@tc##1{\textcolor[rgb]{0.50,0.00,0.50}{##1}}}
\expandafter\def\csname PY@tok@gd\endcsname{\def\PY@tc##1{\textcolor[rgb]{0.63,0.00,0.00}{##1}}}
\expandafter\def\csname PY@tok@gi\endcsname{\def\PY@tc##1{\textcolor[rgb]{0.00,0.63,0.00}{##1}}}
\expandafter\def\csname PY@tok@gr\endcsname{\def\PY@tc##1{\textcolor[rgb]{1.00,0.00,0.00}{##1}}}
\expandafter\def\csname PY@tok@ge\endcsname{\let\PY@it=\textit}
\expandafter\def\csname PY@tok@gs\endcsname{\let\PY@bf=\textbf}
\expandafter\def\csname PY@tok@gp\endcsname{\let\PY@bf=\textbf\def\PY@tc##1{\textcolor[rgb]{0.00,0.00,0.50}{##1}}}
\expandafter\def\csname PY@tok@go\endcsname{\def\PY@tc##1{\textcolor[rgb]{0.53,0.53,0.53}{##1}}}
\expandafter\def\csname PY@tok@gt\endcsname{\def\PY@tc##1{\textcolor[rgb]{0.00,0.27,0.87}{##1}}}
\expandafter\def\csname PY@tok@err\endcsname{\def\PY@bc##1{\setlength{\fboxsep}{0pt}\fcolorbox[rgb]{1.00,0.00,0.00}{1,1,1}{\strut ##1}}}
\expandafter\def\csname PY@tok@kc\endcsname{\let\PY@bf=\textbf\def\PY@tc##1{\textcolor[rgb]{0.00,0.50,0.00}{##1}}}
\expandafter\def\csname PY@tok@kd\endcsname{\let\PY@bf=\textbf\def\PY@tc##1{\textcolor[rgb]{0.00,0.50,0.00}{##1}}}
\expandafter\def\csname PY@tok@kn\endcsname{\let\PY@bf=\textbf\def\PY@tc##1{\textcolor[rgb]{0.00,0.50,0.00}{##1}}}
\expandafter\def\csname PY@tok@kr\endcsname{\let\PY@bf=\textbf\def\PY@tc##1{\textcolor[rgb]{0.00,0.50,0.00}{##1}}}
\expandafter\def\csname PY@tok@bp\endcsname{\def\PY@tc##1{\textcolor[rgb]{0.00,0.50,0.00}{##1}}}
\expandafter\def\csname PY@tok@fm\endcsname{\def\PY@tc##1{\textcolor[rgb]{0.00,0.00,1.00}{##1}}}
\expandafter\def\csname PY@tok@vc\endcsname{\def\PY@tc##1{\textcolor[rgb]{0.10,0.09,0.49}{##1}}}
\expandafter\def\csname PY@tok@vg\endcsname{\def\PY@tc##1{\textcolor[rgb]{0.10,0.09,0.49}{##1}}}
\expandafter\def\csname PY@tok@vi\endcsname{\def\PY@tc##1{\textcolor[rgb]{0.10,0.09,0.49}{##1}}}
\expandafter\def\csname PY@tok@vm\endcsname{\def\PY@tc##1{\textcolor[rgb]{0.10,0.09,0.49}{##1}}}
\expandafter\def\csname PY@tok@sa\endcsname{\def\PY@tc##1{\textcolor[rgb]{0.73,0.13,0.13}{##1}}}
\expandafter\def\csname PY@tok@sb\endcsname{\def\PY@tc##1{\textcolor[rgb]{0.73,0.13,0.13}{##1}}}
\expandafter\def\csname PY@tok@sc\endcsname{\def\PY@tc##1{\textcolor[rgb]{0.73,0.13,0.13}{##1}}}
\expandafter\def\csname PY@tok@dl\endcsname{\def\PY@tc##1{\textcolor[rgb]{0.73,0.13,0.13}{##1}}}
\expandafter\def\csname PY@tok@s2\endcsname{\def\PY@tc##1{\textcolor[rgb]{0.73,0.13,0.13}{##1}}}
\expandafter\def\csname PY@tok@sh\endcsname{\def\PY@tc##1{\textcolor[rgb]{0.73,0.13,0.13}{##1}}}
\expandafter\def\csname PY@tok@s1\endcsname{\def\PY@tc##1{\textcolor[rgb]{0.73,0.13,0.13}{##1}}}
\expandafter\def\csname PY@tok@mb\endcsname{\def\PY@tc##1{\textcolor[rgb]{0.40,0.40,0.40}{##1}}}
\expandafter\def\csname PY@tok@mf\endcsname{\def\PY@tc##1{\textcolor[rgb]{0.40,0.40,0.40}{##1}}}
\expandafter\def\csname PY@tok@mh\endcsname{\def\PY@tc##1{\textcolor[rgb]{0.40,0.40,0.40}{##1}}}
\expandafter\def\csname PY@tok@mi\endcsname{\def\PY@tc##1{\textcolor[rgb]{0.40,0.40,0.40}{##1}}}
\expandafter\def\csname PY@tok@il\endcsname{\def\PY@tc##1{\textcolor[rgb]{0.40,0.40,0.40}{##1}}}
\expandafter\def\csname PY@tok@mo\endcsname{\def\PY@tc##1{\textcolor[rgb]{0.40,0.40,0.40}{##1}}}
\expandafter\def\csname PY@tok@ch\endcsname{\let\PY@it=\textit\def\PY@tc##1{\textcolor[rgb]{0.25,0.50,0.50}{##1}}}
\expandafter\def\csname PY@tok@cm\endcsname{\let\PY@it=\textit\def\PY@tc##1{\textcolor[rgb]{0.25,0.50,0.50}{##1}}}
\expandafter\def\csname PY@tok@cpf\endcsname{\let\PY@it=\textit\def\PY@tc##1{\textcolor[rgb]{0.25,0.50,0.50}{##1}}}
\expandafter\def\csname PY@tok@c1\endcsname{\let\PY@it=\textit\def\PY@tc##1{\textcolor[rgb]{0.25,0.50,0.50}{##1}}}
\expandafter\def\csname PY@tok@cs\endcsname{\let\PY@it=\textit\def\PY@tc##1{\textcolor[rgb]{0.25,0.50,0.50}{##1}}}

\def\PYZbs{\char`\\}
\def\PYZus{\char`\_}
\def\PYZob{\char`\{}
\def\PYZcb{\char`\}}
\def\PYZca{\char`\^}
\def\PYZam{\char`\&}
\def\PYZlt{\char`\<}
\def\PYZgt{\char`\>}
\def\PYZsh{\char`\#}
\def\PYZpc{\char`\%}
\def\PYZdl{\char`\$}
\def\PYZhy{\char`\-}
\def\PYZsq{\char`\'}
\def\PYZdq{\char`\"}
\def\PYZti{\char`\~}
% for compatibility with earlier versions
\def\PYZat{@}
\def\PYZlb{[}
\def\PYZrb{]}
\makeatother


    % Exact colors from NB
    \definecolor{incolor}{rgb}{0.0, 0.0, 0.5}
    \definecolor{outcolor}{rgb}{0.545, 0.0, 0.0}



    
    % Prevent overflowing lines due to hard-to-break entities
    \sloppy 
    % Setup hyperref package
    \hypersetup{
      breaklinks=true,  % so long urls are correctly broken across lines
      colorlinks=true,
      urlcolor=urlcolor,
      linkcolor=linkcolor,
      citecolor=citecolor,
      }
    % Slightly bigger margins than the latex defaults
    
    \geometry{verbose,tmargin=1in,bmargin=1in,lmargin=1in,rmargin=1in}
    
    

    \begin{document}
    
    
    \maketitle
    
    

    
    \hypertarget{machine-learning-engineer-nanodegree}{%
\section{Machine Learning Engineer
Nanodegree}\label{machine-learning-engineer-nanodegree}}

\hypertarget{introduction-and-foundations}{%
\subsection{Introduction and
Foundations}\label{introduction-and-foundations}}

\hypertarget{project-titanic-survival-exploration}{%
\subsection{Project: Titanic Survival
Exploration}\label{project-titanic-survival-exploration}}

In 1912, the ship RMS Titanic struck an iceberg on its maiden voyage and
sank, resulting in the deaths of most of its passengers and crew. In
this introductory project, we will explore a subset of the RMS Titanic
passenger manifest to determine which features best predict whether
someone survived or did not survive. To complete this project, you will
need to implement several conditional predictions and answer the
questions below. Your project submission will be evaluated based on the
completion of the code and your responses to the questions.
\textgreater{} \textbf{Tip:} Quoted sections like this will provide
helpful instructions on how to navigate and use an iPython notebook.

    \hypertarget{getting-started}{%
\section{Getting Started}\label{getting-started}}

To begin working with the RMS Titanic passenger data, we'll first need
to \texttt{import} the functionality we need, and load our data into a
\texttt{pandas} DataFrame.\\
Run the code cell below to load our data and display the first few
entries (passengers) for examination using the \texttt{.head()}
function. \textgreater{} \textbf{Tip:} You can run a code cell by
clicking on the cell and using the keyboard shortcut \textbf{Shift +
Enter} or \textbf{Shift + Return}. Alternatively, a code cell can be
executed using the \textbf{Play} button in the hotbar after selecting
it. Markdown cells (text cells like this one) can be edited by
double-clicking, and saved using these same shortcuts.
\href{http://daringfireball.net/projects/markdown/syntax}{Markdown}
allows you to write easy-to-read plain text that can be converted to
HTML.

    \begin{Verbatim}[commandchars=\\\{\}]
{\color{incolor}In [{\color{incolor}1}]:} \PY{c+c1}{\PYZsh{} Import libraries necessary for this project}
        \PY{k+kn}{import} \PY{n+nn}{numpy} \PY{k}{as} \PY{n+nn}{np}
        \PY{k+kn}{import} \PY{n+nn}{pandas} \PY{k}{as} \PY{n+nn}{pd}
        \PY{k+kn}{from} \PY{n+nn}{IPython}\PY{n+nn}{.}\PY{n+nn}{display} \PY{k}{import} \PY{n}{display} \PY{c+c1}{\PYZsh{} Allows the use of display() for DataFrames}
        
        \PY{c+c1}{\PYZsh{} Import supplementary visualizations code visuals.py}
        \PY{k+kn}{import} \PY{n+nn}{visuals} \PY{k}{as} \PY{n+nn}{vs}
        
        \PY{c+c1}{\PYZsh{} Pretty display for notebooks}
        \PY{o}{\PYZpc{}}\PY{k}{matplotlib} inline
        
        \PY{c+c1}{\PYZsh{} Load the dataset}
        \PY{n}{in\PYZus{}file} \PY{o}{=} \PY{l+s+s1}{\PYZsq{}}\PY{l+s+s1}{titanic\PYZus{}data.csv}\PY{l+s+s1}{\PYZsq{}}
        \PY{n}{full\PYZus{}data} \PY{o}{=} \PY{n}{pd}\PY{o}{.}\PY{n}{read\PYZus{}csv}\PY{p}{(}\PY{n}{in\PYZus{}file}\PY{p}{)}
        
        \PY{c+c1}{\PYZsh{} Print the first few entries of the RMS Titanic data}
        \PY{n}{display}\PY{p}{(}\PY{n}{full\PYZus{}data}\PY{o}{.}\PY{n}{head}\PY{p}{(}\PY{p}{)}\PY{p}{)}
\end{Verbatim}


    
    \begin{verbatim}
   PassengerId  Survived  Pclass  \
0            1         0       3   
1            2         1       1   
2            3         1       3   
3            4         1       1   
4            5         0       3   

                                                Name     Sex   Age  SibSp  \
0                            Braund, Mr. Owen Harris    male  22.0      1   
1  Cumings, Mrs. John Bradley (Florence Briggs Th...  female  38.0      1   
2                             Heikkinen, Miss. Laina  female  26.0      0   
3       Futrelle, Mrs. Jacques Heath (Lily May Peel)  female  35.0      1   
4                           Allen, Mr. William Henry    male  35.0      0   

   Parch            Ticket     Fare Cabin Embarked  
0      0         A/5 21171   7.2500   NaN        S  
1      0          PC 17599  71.2833   C85        C  
2      0  STON/O2. 3101282   7.9250   NaN        S  
3      0            113803  53.1000  C123        S  
4      0            373450   8.0500   NaN        S  
    \end{verbatim}

    
    From a sample of the RMS Titanic data, we can see the various features
present for each passenger on the ship: - \textbf{Survived}: Outcome of
survival (0 = No; 1 = Yes) - \textbf{Pclass}: Socio-economic class (1 =
Upper class; 2 = Middle class; 3 = Lower class) - \textbf{Name}: Name of
passenger - \textbf{Sex}: Sex of the passenger - \textbf{Age}: Age of
the passenger (Some entries contain \texttt{NaN}) - \textbf{SibSp}:
Number of siblings and spouses of the passenger aboard - \textbf{Parch}:
Number of parents and children of the passenger aboard -
\textbf{Ticket}: Ticket number of the passenger - \textbf{Fare}: Fare
paid by the passenger - \textbf{Cabin} Cabin number of the passenger
(Some entries contain \texttt{NaN}) - \textbf{Embarked}: Port of
embarkation of the passenger (C = Cherbourg; Q = Queenstown; S =
Southampton)

Since we're interested in the outcome of survival for each passenger or
crew member, we can remove the \textbf{Survived} feature from this
dataset and store it as its own separate variable \texttt{outcomes}. We
will use these outcomes as our prediction targets.\\
Run the code cell below to remove \textbf{Survived} as a feature of the
dataset and store it in \texttt{outcomes}.

    \begin{Verbatim}[commandchars=\\\{\}]
{\color{incolor}In [{\color{incolor}2}]:} \PY{c+c1}{\PYZsh{} Store the \PYZsq{}Survived\PYZsq{} feature in a new variable and remove it from the dataset}
        \PY{n}{outcomes} \PY{o}{=} \PY{n}{full\PYZus{}data}\PY{p}{[}\PY{l+s+s1}{\PYZsq{}}\PY{l+s+s1}{Survived}\PY{l+s+s1}{\PYZsq{}}\PY{p}{]}
        \PY{n}{data} \PY{o}{=} \PY{n}{full\PYZus{}data}\PY{o}{.}\PY{n}{drop}\PY{p}{(}\PY{l+s+s1}{\PYZsq{}}\PY{l+s+s1}{Survived}\PY{l+s+s1}{\PYZsq{}}\PY{p}{,} \PY{n}{axis} \PY{o}{=} \PY{l+m+mi}{1}\PY{p}{)}
        
        \PY{c+c1}{\PYZsh{} Show the new dataset with \PYZsq{}Survived\PYZsq{} removed}
        \PY{n}{display}\PY{p}{(}\PY{n}{data}\PY{o}{.}\PY{n}{head}\PY{p}{(}\PY{p}{)}\PY{p}{)}
\end{Verbatim}


    
    \begin{verbatim}
   PassengerId  Pclass                                               Name  \
0            1       3                            Braund, Mr. Owen Harris   
1            2       1  Cumings, Mrs. John Bradley (Florence Briggs Th...   
2            3       3                             Heikkinen, Miss. Laina   
3            4       1       Futrelle, Mrs. Jacques Heath (Lily May Peel)   
4            5       3                           Allen, Mr. William Henry   

      Sex   Age  SibSp  Parch            Ticket     Fare Cabin Embarked  
0    male  22.0      1      0         A/5 21171   7.2500   NaN        S  
1  female  38.0      1      0          PC 17599  71.2833   C85        C  
2  female  26.0      0      0  STON/O2. 3101282   7.9250   NaN        S  
3  female  35.0      1      0            113803  53.1000  C123        S  
4    male  35.0      0      0            373450   8.0500   NaN        S  
    \end{verbatim}

    
    The very same sample of the RMS Titanic data now shows the
\textbf{Survived} feature removed from the DataFrame. Note that
\texttt{data} (the passenger data) and \texttt{outcomes} (the outcomes
of survival) are now \emph{paired}. That means for any passenger
\texttt{data.loc{[}i{]}}, they have the survival outcome
\texttt{outcomes{[}i{]}}.

To measure the performance of our predictions, we need a metric to score
our predictions against the true outcomes of survival. Since we are
interested in how \emph{accurate} our predictions are, we will calculate
the proportion of passengers where our prediction of their survival is
correct. Run the code cell below to create our \texttt{accuracy\_score}
function and test a prediction on the first five passengers.

\textbf{Think:} \emph{Out of the first five passengers, if we predict
that all of them survived, what would you expect the accuracy of our
predictions to be?}

    \begin{Verbatim}[commandchars=\\\{\}]
{\color{incolor}In [{\color{incolor}3}]:} \PY{k}{def} \PY{n+nf}{accuracy\PYZus{}score}\PY{p}{(}\PY{n}{truth}\PY{p}{,} \PY{n}{pred}\PY{p}{)}\PY{p}{:}
            \PY{l+s+sd}{\PYZdq{}\PYZdq{}\PYZdq{} Returns accuracy score for input truth and predictions. \PYZdq{}\PYZdq{}\PYZdq{}}
            
            \PY{c+c1}{\PYZsh{} Ensure that the number of predictions matches number of outcomes}
            \PY{k}{if} \PY{n+nb}{len}\PY{p}{(}\PY{n}{truth}\PY{p}{)} \PY{o}{==} \PY{n+nb}{len}\PY{p}{(}\PY{n}{pred}\PY{p}{)}\PY{p}{:} 
                
                \PY{c+c1}{\PYZsh{} Calculate and return the accuracy as a percent}
                \PY{k}{return} \PY{l+s+s2}{\PYZdq{}}\PY{l+s+s2}{Predictions have an accuracy of }\PY{l+s+si}{\PYZob{}:.2f\PYZcb{}}\PY{l+s+s2}{\PYZpc{}}\PY{l+s+s2}{.}\PY{l+s+s2}{\PYZdq{}}\PY{o}{.}\PY{n}{format}\PY{p}{(}\PY{p}{(}\PY{n}{truth} \PY{o}{==} \PY{n}{pred}\PY{p}{)}\PY{o}{.}\PY{n}{mean}\PY{p}{(}\PY{p}{)}\PY{o}{*}\PY{l+m+mi}{100}\PY{p}{)}
            
            \PY{k}{else}\PY{p}{:}
                \PY{k}{return} \PY{l+s+s2}{\PYZdq{}}\PY{l+s+s2}{Number of predictions does not match number of outcomes!}\PY{l+s+s2}{\PYZdq{}}
            
        \PY{c+c1}{\PYZsh{} Test the \PYZsq{}accuracy\PYZus{}score\PYZsq{} function}
        \PY{n}{predictions} \PY{o}{=} \PY{n}{pd}\PY{o}{.}\PY{n}{Series}\PY{p}{(}\PY{n}{np}\PY{o}{.}\PY{n}{ones}\PY{p}{(}\PY{l+m+mi}{5}\PY{p}{,} \PY{n}{dtype} \PY{o}{=} \PY{n+nb}{int}\PY{p}{)}\PY{p}{)}
        \PY{n+nb}{print}\PY{p}{(}\PY{n}{accuracy\PYZus{}score}\PY{p}{(}\PY{n}{outcomes}\PY{p}{[}\PY{p}{:}\PY{l+m+mi}{5}\PY{p}{]}\PY{p}{,} \PY{n}{predictions}\PY{p}{)}\PY{p}{)}
\end{Verbatim}


    \begin{Verbatim}[commandchars=\\\{\}]
Predictions have an accuracy of 60.00\%.

    \end{Verbatim}

    \begin{quote}
\textbf{Tip:} If you save an iPython Notebook, the output from running
code blocks will also be saved. However, the state of your workspace
will be reset once a new session is started. Make sure that you run all
of the code blocks from your previous session to reestablish variables
and functions before picking up where you last left off.
\end{quote}

\hypertarget{making-predictions}{%
\section{Making Predictions}\label{making-predictions}}

If we were asked to make a prediction about any passenger aboard the RMS
Titanic whom we knew nothing about, then the best prediction we could
make would be that they did not survive. This is because we can assume
that a majority of the passengers (more than 50\%) did not survive the
ship sinking.\\
The \texttt{predictions\_0} function below will always predict that a
passenger did not survive.

    \begin{Verbatim}[commandchars=\\\{\}]
{\color{incolor}In [{\color{incolor}4}]:} \PY{k}{def} \PY{n+nf}{predictions\PYZus{}0}\PY{p}{(}\PY{n}{data}\PY{p}{)}\PY{p}{:}
            \PY{l+s+sd}{\PYZdq{}\PYZdq{}\PYZdq{} Model with no features. Always predicts a passenger did not survive. \PYZdq{}\PYZdq{}\PYZdq{}}
        
            \PY{n}{predictions} \PY{o}{=} \PY{p}{[}\PY{p}{]}
            \PY{k}{for} \PY{n}{\PYZus{}}\PY{p}{,} \PY{n}{passenger} \PY{o+ow}{in} \PY{n}{data}\PY{o}{.}\PY{n}{iterrows}\PY{p}{(}\PY{p}{)}\PY{p}{:}
                
                \PY{c+c1}{\PYZsh{} Predict the survival of \PYZsq{}passenger\PYZsq{}}
                \PY{n}{predictions}\PY{o}{.}\PY{n}{append}\PY{p}{(}\PY{l+m+mi}{0}\PY{p}{)}
            
            \PY{c+c1}{\PYZsh{} Return our predictions}
            \PY{k}{return} \PY{n}{pd}\PY{o}{.}\PY{n}{Series}\PY{p}{(}\PY{n}{predictions}\PY{p}{)}
        
        \PY{c+c1}{\PYZsh{} Make the predictions}
        \PY{n}{predictions} \PY{o}{=} \PY{n}{predictions\PYZus{}0}\PY{p}{(}\PY{n}{data}\PY{p}{)}
\end{Verbatim}


    \hypertarget{question-1}{%
\subsubsection{Question 1}\label{question-1}}

\begin{itemize}
\tightlist
\item
  Using the RMS Titanic data, how accurate would a prediction be that
  none of the passengers survived?
\end{itemize}

\textbf{Hint:} Run the code cell below to see the accuracy of this
prediction.

    \begin{Verbatim}[commandchars=\\\{\}]
{\color{incolor}In [{\color{incolor}5}]:} \PY{n+nb}{print}\PY{p}{(}\PY{n}{accuracy\PYZus{}score}\PY{p}{(}\PY{n}{outcomes}\PY{p}{,} \PY{n}{predictions}\PY{p}{)}\PY{p}{)}
\end{Verbatim}


    \begin{Verbatim}[commandchars=\\\{\}]
Predictions have an accuracy of 61.62\%.

    \end{Verbatim}

    \textbf{Answer:} Predictions have an accuracy of 61.62\%.

    \begin{center}\rule{0.5\linewidth}{\linethickness}\end{center}

Let's take a look at whether the feature \textbf{Sex} has any indication
of survival rates among passengers using the \texttt{survival\_stats}
function. This function is defined in the \texttt{visuals.py} Python
script included with this project. The first two parameters passed to
the function are the RMS Titanic data and passenger survival outcomes,
respectively. The third parameter indicates which feature we want to
plot survival statistics across.\\
Run the code cell below to plot the survival outcomes of passengers
based on their sex.

    \begin{Verbatim}[commandchars=\\\{\}]
{\color{incolor}In [{\color{incolor}6}]:} \PY{n}{vs}\PY{o}{.}\PY{n}{survival\PYZus{}stats}\PY{p}{(}\PY{n}{data}\PY{p}{,} \PY{n}{outcomes}\PY{p}{,} \PY{l+s+s1}{\PYZsq{}}\PY{l+s+s1}{Sex}\PY{l+s+s1}{\PYZsq{}}\PY{p}{)}
\end{Verbatim}


    \begin{center}
    \adjustimage{max size={0.9\linewidth}{0.9\paperheight}}{output_13_0.png}
    \end{center}
    { \hspace*{\fill} \\}
    
    Examining the survival statistics, a large majority of males did not
survive the ship sinking. However, a majority of females \emph{did}
survive the ship sinking. Let's build on our previous prediction: If a
passenger was female, then we will predict that they survived.
Otherwise, we will predict the passenger did not survive.\\
Fill in the missing code below so that the function will make this
prediction.\\
\textbf{Hint:} You can access the values of each feature for a passenger
like a dictionary. For example,
\texttt{passenger{[}\textquotesingle{}Sex\textquotesingle{}{]}} is the
sex of the passenger.

    \begin{Verbatim}[commandchars=\\\{\}]
{\color{incolor}In [{\color{incolor}7}]:} \PY{k}{def} \PY{n+nf}{predictions\PYZus{}1}\PY{p}{(}\PY{n}{data}\PY{p}{)}\PY{p}{:}
            \PY{l+s+sd}{\PYZdq{}\PYZdq{}\PYZdq{} Model with one feature: }
        \PY{l+s+sd}{            \PYZhy{} Predict a passenger survived if they are female. \PYZdq{}\PYZdq{}\PYZdq{}}
            
            \PY{n}{predictions} \PY{o}{=} \PY{p}{[}\PY{p}{]}
            \PY{k}{for} \PY{n}{\PYZus{}}\PY{p}{,} \PY{n}{passenger} \PY{o+ow}{in} \PY{n}{data}\PY{o}{.}\PY{n}{iterrows}\PY{p}{(}\PY{p}{)}\PY{p}{:}
                
                \PY{c+c1}{\PYZsh{} Remove the \PYZsq{}pass\PYZsq{} statement below }
                \PY{c+c1}{\PYZsh{} and write your prediction conditions here}
                \PY{k}{if} \PY{n}{passenger}\PY{p}{[}\PY{l+s+s1}{\PYZsq{}}\PY{l+s+s1}{Sex}\PY{l+s+s1}{\PYZsq{}}\PY{p}{]} \PY{o}{==} \PY{l+s+s1}{\PYZsq{}}\PY{l+s+s1}{female}\PY{l+s+s1}{\PYZsq{}}\PY{p}{:}
                    \PY{n}{predictions}\PY{o}{.}\PY{n}{append}\PY{p}{(}\PY{l+m+mi}{1}\PY{p}{)}
                \PY{k}{else}\PY{p}{:}
                    \PY{n}{predictions}\PY{o}{.}\PY{n}{append}\PY{p}{(}\PY{l+m+mi}{0}\PY{p}{)}        
            
            \PY{c+c1}{\PYZsh{} Return our predictions}
            \PY{k}{return} \PY{n}{pd}\PY{o}{.}\PY{n}{Series}\PY{p}{(}\PY{n}{predictions}\PY{p}{)}
        
        \PY{c+c1}{\PYZsh{} Make the predictions}
        \PY{n}{predictions} \PY{o}{=} \PY{n}{predictions\PYZus{}1}\PY{p}{(}\PY{n}{data}\PY{p}{)}
\end{Verbatim}


    \hypertarget{question-2}{%
\subsubsection{Question 2}\label{question-2}}

\begin{itemize}
\tightlist
\item
  How accurate would a prediction be that all female passengers survived
  and the remaining passengers did not survive?
\end{itemize}

\textbf{Hint:} Run the code cell below to see the accuracy of this
prediction.

    \begin{Verbatim}[commandchars=\\\{\}]
{\color{incolor}In [{\color{incolor}8}]:} \PY{n+nb}{print}\PY{p}{(}\PY{n}{accuracy\PYZus{}score}\PY{p}{(}\PY{n}{outcomes}\PY{p}{,} \PY{n}{predictions}\PY{p}{)}\PY{p}{)}
\end{Verbatim}


    \begin{Verbatim}[commandchars=\\\{\}]
Predictions have an accuracy of 78.68\%.

    \end{Verbatim}

    \textbf{Answer}: Predictions have an accuracy of 78.68\%.

    \begin{center}\rule{0.5\linewidth}{\linethickness}\end{center}

Using just the \textbf{Sex} feature for each passenger, we are able to
increase the accuracy of our predictions by a significant margin. Now,
let's consider using an additional feature to see if we can further
improve our predictions. For example, consider all of the male
passengers aboard the RMS Titanic: Can we find a subset of those
passengers that had a higher rate of survival? Let's start by looking at
the \textbf{Age} of each male, by again using the
\texttt{survival\_stats} function. This time, we'll use a fourth
parameter to filter out the data so that only passengers with the
\textbf{Sex} `male' will be included.\\
Run the code cell below to plot the survival outcomes of male passengers
based on their age.

    \begin{Verbatim}[commandchars=\\\{\}]
{\color{incolor}In [{\color{incolor}9}]:} \PY{n}{vs}\PY{o}{.}\PY{n}{survival\PYZus{}stats}\PY{p}{(}\PY{n}{data}\PY{p}{,} \PY{n}{outcomes}\PY{p}{,} \PY{l+s+s1}{\PYZsq{}}\PY{l+s+s1}{Age}\PY{l+s+s1}{\PYZsq{}}\PY{p}{,} \PY{p}{[}\PY{l+s+s2}{\PYZdq{}}\PY{l+s+s2}{Sex == }\PY{l+s+s2}{\PYZsq{}}\PY{l+s+s2}{female}\PY{l+s+s2}{\PYZsq{}}\PY{l+s+s2}{\PYZdq{}}\PY{p}{,} \PY{l+s+s2}{\PYZdq{}}\PY{l+s+s2}{Pclass == 3}\PY{l+s+s2}{\PYZdq{}}\PY{p}{]}\PY{p}{)}
\end{Verbatim}


    \begin{center}
    \adjustimage{max size={0.9\linewidth}{0.9\paperheight}}{output_20_0.png}
    \end{center}
    { \hspace*{\fill} \\}
    
    Examining the survival statistics, the majority of males younger than 10
survived the ship sinking, whereas most males age 10 or older \emph{did
not survive} the ship sinking. Let's continue to build on our previous
prediction: If a passenger was female, then we will predict they
survive. If a passenger was male and younger than 10, then we will also
predict they survive. Otherwise, we will predict they do not survive.\\
Fill in the missing code below so that the function will make this
prediction.\\
\textbf{Hint:} You can start your implementation of this function using
the prediction code you wrote earlier from \texttt{predictions\_1}.

    \begin{Verbatim}[commandchars=\\\{\}]
{\color{incolor}In [{\color{incolor}10}]:} \PY{k}{def} \PY{n+nf}{predictions\PYZus{}2}\PY{p}{(}\PY{n}{data}\PY{p}{)}\PY{p}{:}
             \PY{l+s+sd}{\PYZdq{}\PYZdq{}\PYZdq{} Model with two features: }
         \PY{l+s+sd}{            \PYZhy{} Predict a passenger survived if they are female.}
         \PY{l+s+sd}{            \PYZhy{} Predict a passenger survived if they are male and younger than 10. \PYZdq{}\PYZdq{}\PYZdq{}}
             
             \PY{n}{predictions} \PY{o}{=} \PY{p}{[}\PY{p}{]}
             \PY{k}{for} \PY{n}{\PYZus{}}\PY{p}{,} \PY{n}{passenger} \PY{o+ow}{in} \PY{n}{data}\PY{o}{.}\PY{n}{iterrows}\PY{p}{(}\PY{p}{)}\PY{p}{:}
                 
                 \PY{c+c1}{\PYZsh{} Remove the \PYZsq{}pass\PYZsq{} statement below }
                 \PY{c+c1}{\PYZsh{} and write your prediction conditions here}
                 \PY{k}{if} \PY{n}{passenger}\PY{p}{[}\PY{l+s+s1}{\PYZsq{}}\PY{l+s+s1}{Sex}\PY{l+s+s1}{\PYZsq{}}\PY{p}{]} \PY{o}{==} \PY{l+s+s1}{\PYZsq{}}\PY{l+s+s1}{female}\PY{l+s+s1}{\PYZsq{}}\PY{p}{:}
                     \PY{n}{predictions}\PY{o}{.}\PY{n}{append}\PY{p}{(}\PY{l+m+mi}{1}\PY{p}{)}
                 \PY{k}{elif} \PY{n}{passenger}\PY{p}{[}\PY{l+s+s1}{\PYZsq{}}\PY{l+s+s1}{Sex}\PY{l+s+s1}{\PYZsq{}}\PY{p}{]} \PY{o}{==} \PY{l+s+s1}{\PYZsq{}}\PY{l+s+s1}{male}\PY{l+s+s1}{\PYZsq{}} \PY{o+ow}{and} \PY{n}{passenger}\PY{p}{[}\PY{l+s+s1}{\PYZsq{}}\PY{l+s+s1}{Age}\PY{l+s+s1}{\PYZsq{}}\PY{p}{]} \PY{o}{\PYZlt{}} \PY{l+m+mi}{10}\PY{p}{:}
                     \PY{n}{predictions}\PY{o}{.}\PY{n}{append}\PY{p}{(}\PY{l+m+mi}{1}\PY{p}{)}
                 \PY{k}{else}\PY{p}{:}
                     \PY{n}{predictions}\PY{o}{.}\PY{n}{append}\PY{p}{(}\PY{l+m+mi}{0}\PY{p}{)}
             
             \PY{c+c1}{\PYZsh{} Return our predictions}
             \PY{k}{return} \PY{n}{pd}\PY{o}{.}\PY{n}{Series}\PY{p}{(}\PY{n}{predictions}\PY{p}{)}
         
         \PY{c+c1}{\PYZsh{} Make the predictions}
         \PY{n}{predictions} \PY{o}{=} \PY{n}{predictions\PYZus{}2}\PY{p}{(}\PY{n}{data}\PY{p}{)}
\end{Verbatim}


    \hypertarget{question-3}{%
\subsubsection{Question 3}\label{question-3}}

\begin{itemize}
\tightlist
\item
  How accurate would a prediction be that all female passengers and all
  male passengers younger than 10 survived?
\end{itemize}

\textbf{Hint:} Run the code cell below to see the accuracy of this
prediction.

    \begin{Verbatim}[commandchars=\\\{\}]
{\color{incolor}In [{\color{incolor}11}]:} \PY{n+nb}{print}\PY{p}{(}\PY{n}{accuracy\PYZus{}score}\PY{p}{(}\PY{n}{outcomes}\PY{p}{,} \PY{n}{predictions}\PY{p}{)}\PY{p}{)}
\end{Verbatim}


    \begin{Verbatim}[commandchars=\\\{\}]
Predictions have an accuracy of 79.35\%.

    \end{Verbatim}

    \hypertarget{answer-predictions-have-an-accuracy-of-79.24.}{%
\subparagraph{\texorpdfstring{\textbf{Answer}: Predictions have an
accuracy of
79.24\%.}{Answer: Predictions have an accuracy of 79.24\%.}}\label{answer-predictions-have-an-accuracy-of-79.24.}}

    \begin{center}\rule{0.5\linewidth}{\linethickness}\end{center}

Adding the feature \textbf{Age} as a condition in conjunction with
\textbf{Sex} improves the accuracy by a small margin more than with
simply using the feature \textbf{Sex} alone. Now it's your turn: Find a
series of features and conditions to split the data on to obtain an
outcome prediction accuracy of at least 80\%. This may require multiple
features and multiple levels of conditional statements to succeed. You
can use the same feature multiple times with different conditions.\\
\textbf{Pclass}, \textbf{Sex}, \textbf{Age}, \textbf{SibSp}, and
\textbf{Parch} are some suggested features to try.

Use the \texttt{survival\_stats} function below to to examine various
survival statistics.\\
\textbf{Hint:} To use mulitple filter conditions, put each condition in
the list passed as the last argument. Example:
\texttt{{[}"Sex\ ==\ \textquotesingle{}male\textquotesingle{}",\ "Age\ \textless{}\ 18"{]}}

    \begin{Verbatim}[commandchars=\\\{\}]
{\color{incolor}In [{\color{incolor}12}]:} \PY{n}{vs}\PY{o}{.}\PY{n}{survival\PYZus{}stats}\PY{p}{(}\PY{n}{data}\PY{p}{,} \PY{n}{outcomes}\PY{p}{,} \PY{l+s+s1}{\PYZsq{}}\PY{l+s+s1}{Age}\PY{l+s+s1}{\PYZsq{}}\PY{p}{,} \PY{p}{[}\PY{l+s+s2}{\PYZdq{}}\PY{l+s+s2}{Sex == }\PY{l+s+s2}{\PYZsq{}}\PY{l+s+s2}{female}\PY{l+s+s2}{\PYZsq{}}\PY{l+s+s2}{\PYZdq{}}\PY{p}{,} \PY{l+s+s2}{\PYZdq{}}\PY{l+s+s2}{Pclass == 3}\PY{l+s+s2}{\PYZdq{}}\PY{p}{]}\PY{p}{)}
\end{Verbatim}


    \begin{center}
    \adjustimage{max size={0.9\linewidth}{0.9\paperheight}}{output_27_0.png}
    \end{center}
    { \hspace*{\fill} \\}
    
    After exploring the survival statistics visualization, fill in the
missing code below so that the function will make your prediction.\\
Make sure to keep track of the various features and conditions you tried
before arriving at your final prediction model.\\
\textbf{Hint:} You can start your implementation of this function using
the prediction code you wrote earlier from \texttt{predictions\_2}.

    \begin{Verbatim}[commandchars=\\\{\}]
{\color{incolor}In [{\color{incolor}13}]:} \PY{k}{def} \PY{n+nf}{predictions\PYZus{}3}\PY{p}{(}\PY{n}{data}\PY{p}{)}\PY{p}{:}
             \PY{l+s+sd}{\PYZdq{}\PYZdq{}\PYZdq{} Model with multiple features. Makes a prediction with an accuracy of at least 80\PYZpc{}. \PYZdq{}\PYZdq{}\PYZdq{}}
             
             \PY{n}{predictions} \PY{o}{=} \PY{p}{[}\PY{p}{]}
             \PY{k}{for} \PY{n}{\PYZus{}}\PY{p}{,} \PY{n}{passenger} \PY{o+ow}{in} \PY{n}{data}\PY{o}{.}\PY{n}{iterrows}\PY{p}{(}\PY{p}{)}\PY{p}{:}
                 
                 \PY{c+c1}{\PYZsh{} Remove the \PYZsq{}pass\PYZsq{} statement below }
                 \PY{c+c1}{\PYZsh{} and write your prediction conditions here}
                 \PY{c+c1}{\PYZsh{} Remove the \PYZsq{}pass\PYZsq{} statement below }
                 \PY{c+c1}{\PYZsh{} and write your prediction conditions here}
                 \PY{k}{if} \PY{n}{passenger}\PY{p}{[}\PY{l+s+s1}{\PYZsq{}}\PY{l+s+s1}{Sex}\PY{l+s+s1}{\PYZsq{}}\PY{p}{]} \PY{o}{==} \PY{l+s+s1}{\PYZsq{}}\PY{l+s+s1}{female}\PY{l+s+s1}{\PYZsq{}}\PY{p}{:}
                     \PY{k}{if} \PY{n}{passenger}\PY{p}{[}\PY{l+s+s1}{\PYZsq{}}\PY{l+s+s1}{Age}\PY{l+s+s1}{\PYZsq{}}\PY{p}{]} \PY{o}{\PYZgt{}} \PY{l+m+mi}{40} \PY{o+ow}{and} \PY{n}{passenger}\PY{p}{[}\PY{l+s+s1}{\PYZsq{}}\PY{l+s+s1}{Age}\PY{l+s+s1}{\PYZsq{}}\PY{p}{]} \PY{o}{\PYZlt{}}\PY{o}{=} \PY{l+m+mi}{50} \PY{o+ow}{and} \PY{n}{passenger}\PY{p}{[}\PY{l+s+s1}{\PYZsq{}}\PY{l+s+s1}{Pclass}\PY{l+s+s1}{\PYZsq{}}\PY{p}{]} \PY{o}{==} \PY{l+m+mi}{3}\PY{p}{:}
                         \PY{n}{predictions}\PY{o}{.}\PY{n}{append}\PY{p}{(}\PY{l+m+mi}{0}\PY{p}{)}
                     \PY{k}{else}\PY{p}{:}
                         \PY{n}{predictions}\PY{o}{.}\PY{n}{append}\PY{p}{(}\PY{l+m+mi}{1}\PY{p}{)}
                 \PY{k}{elif} \PY{n}{passenger}\PY{p}{[}\PY{l+s+s1}{\PYZsq{}}\PY{l+s+s1}{Sex}\PY{l+s+s1}{\PYZsq{}}\PY{p}{]} \PY{o}{==} \PY{l+s+s1}{\PYZsq{}}\PY{l+s+s1}{male}\PY{l+s+s1}{\PYZsq{}} \PY{o+ow}{and} \PY{n}{passenger}\PY{p}{[}\PY{l+s+s1}{\PYZsq{}}\PY{l+s+s1}{Age}\PY{l+s+s1}{\PYZsq{}}\PY{p}{]} \PY{o}{\PYZlt{}}\PY{o}{=} \PY{l+m+mi}{10}\PY{p}{:}
                     \PY{n}{predictions}\PY{o}{.}\PY{n}{append}\PY{p}{(}\PY{l+m+mi}{1}\PY{p}{)}
                 \PY{k}{else}\PY{p}{:}
                     \PY{n}{predictions}\PY{o}{.}\PY{n}{append}\PY{p}{(}\PY{l+m+mi}{0}\PY{p}{)}
         
             \PY{c+c1}{\PYZsh{} Return our predictions}
             \PY{k}{return} \PY{n}{pd}\PY{o}{.}\PY{n}{Series}\PY{p}{(}\PY{n}{predictions}\PY{p}{)}
         
         \PY{c+c1}{\PYZsh{} Make the predictions}
         \PY{n}{predictions} \PY{o}{=} \PY{n}{predictions\PYZus{}3}\PY{p}{(}\PY{n}{data}\PY{p}{)}
\end{Verbatim}


    \hypertarget{question-4}. What features did
  you look at? Were certain features more informative than others? Which
  conditions did you use to split the survival outcomes in the data? How
  accurate are your predictions?
\end{itemize}

\textbf{Hint:} Run the code cell below to see the accuracy of your
predictions.

    \begin{Verbatim}[commandchars=\\\{\}]
{\color{incolor}In [{\color{incolor}14}]:} \PY{n+nb}{print}\PY{p}{(}\PY{n}{accuracy\PYZus{}score}\PY{p}{(}\PY{n}{outcomes}\PY{p}{,} \PY{n}{predictions}\PY{p}{)}\PY{p}{)}
\end{Verbatim}


    \begin{Verbatim}[commandchars=\\\{\}]
Predictions have an accuracy of 80.13\%.

    \end{Verbatim}

    \textbf{Answer}: For females, I looked at survival by age for passenger
class `Pclass' and Sibling/Spouse `SibSp' features. The `SibSp' did not
seem to improve the accuracy. Using the passenger class feature `Pclass'
was more informative for third class passengers than first or second
class. It showed a very distinct none-survival rate for females aged
between 40 and 50 years old in third passenger class, which is quite
different from other ages and other classes. With that, I obtained an
accuracy of 80.13\%.

    \hypertarget{conclusion}{%
\section{Conclusion}\label{conclusion}}

After several iterations of exploring and conditioning on the data, you
have built a useful algorithm for predicting the survival of each
passenger aboard the RMS Titanic. The technique applied in this project
is a manual implementation of a simple machine learning model, the
\emph{decision tree}. A decision tree splits a set of data into smaller
and smaller groups (called \emph{nodes}), by one feature at a time. Each
time a subset of the data is split, our predictions become more accurate
if each of the resulting subgroups are more homogeneous (contain similar
labels) than before. The advantage of having a computer do things for us
is that it will be more exhaustive and more precise than our manual
exploration above.
\href{http://www.r2d3.us/visual-intro-to-machine-learning-part-1/}{This
link} provides another introduction into machine learning using a
decision tree.

A decision tree is just one of many models that come from
\emph{supervised learning}. In supervised learning, we attempt to use
features of the data to predict or model things with objective outcome
labels. That is to say, each of our data points has a known outcome
value, such as a categorical, discrete label like
\texttt{\textquotesingle{}Survived\textquotesingle{}}, or a numerical,
continuous value like predicting the price of a house.

\hypertarget{question-5}{%
\subsubsection{Question 5}\label{question-5}}

\emph{Think of a real-world scenario where supervised learning could be
applied. What would be the outcome variable that you are trying to
predict? Name two features about the data used in this scenario that
might be helpful for making the predictions.}

    \textbf{Answer}: Record average yearly temperature in a tropical village
over the last 40 years. Measure the age, economic class, and number of
times ill from infectious disease. Try to predict the incidence of
infectious disease as temperatures rise.

    \begin{quote}
\textbf{Note}: Once you have completed all of the code implementations
and successfully answered each question above, you may finalize your
work by exporting the iPython Notebook as an HTML document. You can do
this by using the menu above and navigating to\\
\textbf{File -\textgreater{} Download as -\textgreater{} HTML (.html)}.
Include the finished document along with this notebook as your
submission.
\end{quote}


    % Add a bibliography block to the postdoc
    
    
    
    \end{document}
